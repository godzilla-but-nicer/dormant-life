\documentclass[12pt]{article}
\usepackage[margin=1in]{geometry}

% Refs
\usepackage[style=nature, backend=bibtex]{biblatex}
\addbibresource{dormant-life.bib}
\usepackage[usenames,dvipsnames,svgnames,table]{xcolor}

% Math
\usepackage{amsmath}
\usepackage{amssymb}

% Figures
\usepackage{graphicx}
\newcommand{\figref}[1]{Fig.~\ref{fig:#1}}

% subfigures
\usepackage{caption}
\usepackage{subcaption}

% author notes
\newcommand{\note}[1]{\textcolor{red}{[#1]}}

\title{Dormancy Produces Long Transients in Game of Life}
\author{Pat Wall}
\date{\today}


\begin{document}

\maketitle

\section{Long Transients in DormantLife}

Transient states are those that a system passes through on its way to an
attractor or limit-cycle steady state. The number of such transient states
visited by a system has long been used as a measure of system complexity with
long transients indicating greater complexity
\cite{wolfram_statistical_1983,langton_computation_1990}. Such transient
dynamics have long been implicated is potentially important in the overall
dynamics of ecological communities
\cite{hastings_transient_2018,turchin_complex_1992}.

We simulated Conway's Game of Life, as well as the modified DormantLife
\cite{hutchison_cell_2006} on a 7x7 cell grid starting from 10000 uniformly
distributed initial system states. Each simulation was halted was stopped after
the system entered any state for the second time. The number of states in
between the two occurrences of the repeated state defined the period ($T$) while
the number of states from initialization to the first occurrence of the repeated
state defined the transient length ($l$). 

The addition of the dormant state to Game of Life resulted in longer transient
lengths by a factor of nearly 100 on average as shown in
\figref{transient-hist}.

\begin{figure}
    \centering
    \includegraphics[width=0.75\textwidth]{assets/transient_dists_77.pdf}
    \caption{\textbf{Dormancy increases transient lengths in Game of Life.}
    Histogram of log-scale transient lengths for Game of Life (blue) and
    DormantLife (orange) with dashed lines showing the mean transient length for
    the two models. The mean for DormantLife is nearly 100x larger than the mean
    for Game of Life.}
    \label{fig:transient-hist}
\end{figure}

\section{More Living Cells in DormantLife}

Game of Life and the DormantLife model both consider both living and unoccupied
or dead cells. We counted the number of living cells for the first 50 time
steps of both Game of Life and DormantLife in a 10x10 grid. For the sake of
calculating this number we did not consider cells in the dormant state to be
alive in the DormantLife model.

We found that DormantLife produces a significantly higher number of living cells
after an initial settling period of about 4 time steps. \figref{living-cells}
shows DormantLife approaching a positive number of living cells while Game of
Life continually decreases.

\begin{figure}
    \centering
    \includegraphics[width=0.75\textwidth]{assets/livingcells.pdf}
    \caption{\textbf{Dormancy increases the number of cells that are alive in
    Game of Life.} Average number of living cells in the first 50 time steps of
    randomly initialized Life models with bootstrap confidence intervals.
    DormantLife produces consistently greater numbers of living cells than
    Game of Life.}
    \label{fig:living-cells}
\end{figure}

We can estimate the number of cells alive at equilibrium by considering an
ensemble of neighborhoods for our life simulations occupying all possible
neighborhood states in equal proportion. \note{i think this assumes argodicity
which is definitely not true} Then we can consider the next step of each of
these replicate neighborhoods. We can define the probability of a cell being
in the living state as



\begin{align}
    P_n = P_{n-1}l + (1 - P_{n-1})k\\
    P_n = P_{n-1} l + k - P_{n - 1} k
\end{align}

\printbibliography
\end{document}
